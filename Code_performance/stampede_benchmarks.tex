\documentclass[11pt]{article}

\usepackage{amssymb}
\usepackage{amsmath}
\usepackage{dsfont}
\usepackage{graphicx}
\usepackage{multirow}
\usepackage{hyperref}
\usepackage{color}
\usepackage{pbox}

\addtolength{\hoffset}{-2.5cm}
\addtolength{\textwidth}{4.5cm}
\addtolength{\voffset}{-2cm}
\addtolength{\textheight}{4.5cm}

\begin{document}

\section*{Inspector Gadget Pipeline 0.1: Stampede performance}

\begin{table}[h!]
\begin{center}
\begin{tabular}{c|c|c|c|c}
\textbf{Step} & \textbf{Test performed} & \textbf{Runtime/cores used} & \textbf{Cost} & \textbf{Rough scaling} \\ \hline
\pbox{20cm}{Matter power spectra \\ (CAMB)} & \pbox{20cm}{16 power spectra \\ in parallel} & 1h 16m 51s/16 & 17.8 SU & 1.11 SU $\times N_{spectra}$ \\ \hline
\pbox{20cm}{Initial conditions\\ generation (N-GenIC)} & \pbox{20cm}{1 initial condition \\ ($512^3$ particles, 240 Mpc)} & 27s/256 & 1.92 SU & 1.92 SU $\times N_{ICs,512}$ \\ \hline
\pbox{20cm}{Initial conditions\\ generation (N-GenIC)} & \pbox{20cm}{1 initial condition \\ ($1024^3$ particles, 480 MPc)} & 47s/512 & 6.68 SU & 6.68 SU $\times N_{ICs,1024}$ \\ \hline
\pbox{20cm}{$N$-body simulations\\ (Gadget2)} & \pbox{20cm}{1 simulation \\ ($512^3$ particles)} & 3h 3m 16s/256 & 781.9 SU & 781.9 SU $\times N_{Sims}$ \\ \hline
2D plane projections & \pbox{20cm}{Creation of 9 lens planes \\ per snapshot (58 snapshots,\\ 1 simulation)} & 7m 15s/60 & 7.7 SU & - \\ \hline
2D plane projections & \pbox{20cm}{Creation of 9 lens planes \\ per snapshot (58 snapshots,\\ 4 simulations)} & 6m 59s/240 & 27.9 SU & \pbox{20cm}{0.11 SU$\times N_{snapshots}$ \\ $\times N_{Sims}$} \\ \hline
\pbox{20cm}{Ray tracing \\ (CFHT catalogs)} & \pbox{20cm}{2 subfields $\times$ 32 realizations \\ (from 1 simulation only \\with plane \\preloading in RAM)} & 9m 36s/64 & 10.24 SU & \pbox{20cm}{0.16 SU $\times N_{subfields}$\\ $\times N_{realizations}$\\ (not including \\ I/O corrections)} \\ \hline
\pbox{20cm}{Ray tracing \\ (CFHT catalogs)} & \pbox{20cm}{2 subfields $\times$ 256 realizations \\ (mixing 5 simulations \\with plane \\preloading in RAM)} & 10m 45s/512 & 91.73 SU & \pbox{20cm}{0.18 SU $\times N_{subfields}$\\ $\times N_{realizations}$\\ (not including \\ I/O corrections)} \\ \hline
\pbox{20cm}{Ray tracing \\(CFHT catalogs)} & \pbox{20cm}{2 subfields $\times$ 256 realizations\\ (mixing 5 simulations\\no plane preloading, \\more I/O intensive)} & 14m 06s/256 & 60.16 SU & \pbox{20cm}{0.12 SU $\times N_{subfields}$\\ $\times N_{realizations}$\\ (not including \\ I/O corrections)} \\ \hline
\pbox{20cm}{Ray tracing\\ ($\kappa,\gamma$ maps)} & \pbox{20cm}{32 realizations $(\kappa,\gamma)$ \\ (1 simulation only, \\ with plane preloading)} & 16m 50s/64 & 17.9 SU & \pbox{20cm}{0.55 SU $\times N_{realizations}$\\ (not including \\ I/O corrections)} \\ \hline
\end{tabular}
\end{center}
\caption{\textit{"Not including I/O corrections"} means that, based on the performance of the scratch disk/memory, the computing cost might scale a little worse than linear when we use more cores to read simulteneously from disk. I can run additional tests if required. Note rows 7 and 8: compare the rates of SUs burned per subfield per realization (that should be independent on the number of processors if we neglect I/O, because the code is designed so that to producing $n$ maps with $m$ processors should scale exacly as $n/m$). In reality there is a little tilt in the cost scaling with the number of processors $m^\alpha$ where $\alpha$ quantifies how much the I/O and communications are non ideal; with 2 data points I can give you a number to throw in the proposal $\alpha\sim\frac{\log 0.18 - \log 0.16}{\log 512 - \log 64}\sim 0.057$. I think we are OK becuse this is really small}
\end{table}

\end{document}